\documentclass{article}
\usepackage[utf8]{inputenc}
\usepackage{amsmath}
\usepackage{amssymb}
\usepackage{amsthm}

\title{Some Elementary Notes on the String Moduli Space}

\author{Jake Bian}

\date{Jan 2016}


\begin{document}

    \maketitle

    \abstract{
        I explain some aspects of the geometric setting for string theory.
    }


    \section{The Classical String Moduli Space}

        \subsection{The Moduli Space}
        \paragraph{The space of all embeddings}

            Let $\Sigma$ be a Riemann surface. Let $(M, g_M)$ be a pseudo-Riemannian manifold of dimendion $d$.

            We are interested in the ways in which one can smoothly embed $\Sigma$ into $M$. Naively  is, we are interested in the space
            \begin{equation}
                E \equiv \{ \text{smooth embeddings } e: \Sigma \ to M \}
            \end{equation}

            But note that this really isn't quite what we're looking for. This is because one can pre-compose an embedding $e$ with an automorphism $f: \Sigma \to \Sigma$. The composed map $e \circ f \in E$ is another smooth embedding of $\Sigma$ in $M$, but their images are identical

            \begin{equation}
                e(\Sigma) = e(f(\Sigma)) \subset M
            \end{equation}

            So they really describe the same embedding of $\Sigma$ in $M$. So in order to define a less redundant notion of ``the space of all embeddings'', we need to quotient our space by these precompositions

            \begin{equation}
                \tilde E \equiv E/\sim ~,~ e_1 \sim e_2 \iff e_2 = e_1 \circ f \text{ for some } f \in Aut(\Sigma)
            \end{equation}

            This is the notion of the moduli space in the classical theory of a bosonic string.

        \paragraph{The space of induced metrics} 

            From the space $\tilde E$ all allowed embeddings, we would now like to select special elements which minimize a certain functional. One might write this functional as $S[e]$ for $e \in \tilde E$.

            A simple functional one might want to consider is one which computs the area of the surface $e(\Sigma) \subset M$ under the Riemannian structure on $M$. One then immediately realizes that writing the area formula in terms of the embedding map $e$ is cumbersome. Instead it is much easier to write it in terms of the induced metric on the surface by the embedding. So instead of looking at the space of maps $\tilde E$, we want to consider the space of induced metrics

            \begin{equation}
                \tilde F \equiv \{e^* g_M | e \in \tilde E \} \subset T\Sigma^{\otimes 2}
            \end{equation}

            Recall we obtained $\tilde E$ as a quotient. It is useful to know how this quotient structure manifests in the space $\tilde F$. Recall that the de Rham functor is contravariant

            \begin{equation}
                (e \circ f)^* = f^* \circ e^*
            \end{equation}

            Hence we can write

            \begin{align}
                F &\equiv \{ e^*g_M | e \in E \}\\
                \tilde F &\equiv F / \sim ~,~ h_1 \sim h_2 \iff h_1 = f^* h_2 \text{ for some } f \in Aut(\Sigma)
            \end{align}

        \subsection{Gauge Fixing}
            \paragraph{The general story}

                Recall again that our moduli space has is a quotient

                \begin{equation}
                    \tilde E = E / \sim
                \end{equation}

                Gauge fixing means smoothly choosing a representative for each equivalence class in this quotient In other words, we want a smooth section $s$ with respect to the quotient map $\pi: E \to \tilde E$.

                The reason we want to do this is to be able to explicitly perform the variational problem for the functional $S[e]$. $S[e]$ is straightforward to write down as a problem with the domain being $E$. In order to reduce the variational problem to the quotiented domain of maps $\tilde E$, one must choose representatives from each equivalence class $[e]$. We would also like this choice of representatives to be smooth, so that one can actually consider the variations of these representatives.

                Given a ``gauge choice'' $s: \tilde E \to E$ as described above, one can simply considers the variational problem with the domain $s(\tilde E)$.

            \subsubsection{Examples}
                \paragraph{Static Gauge}
                    Let $\sigma, \tau$ be local coordinates on $\Sigma$. For each such local chart, one defines the following.

                    Let $[e] \in \tilde E$, define $e_S \in [e]$ as the embedding such that $\forall p \in M$ $e_S^0|_p = \tau|_p$.


                    Define then the gauge choice $s$ to be 
                    \begin{equation}
                        s: \tilde E \to E, ~~ [e] \mapsto e_S
                    \end{equation}


\end{document}
