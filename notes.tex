\documentclass{article}
\usepackage[utf8]{inputenc}
\usepackage{amsmath}
\usepackage{amssymb}
\usepackage{amsthm}

\title{Some Notes on String Theory}

\author{Jake Bian}

\date{Jan 2016}


\begin{document}

    \maketitle

    \abstract{
        These notes aim to clarify the geometric setting of some constructions in string theory.
    }


    \section{The Classical String Moduli Space}

        \subsection{The Moduli Space}
        \paragraph{The space of all embeddings}

            Let $\Sigma$ be a Riemann surface. Let $(M, g_M)$ be a pseudo-Riemannian manifold of dimendion $d$.

            We are interested in the ways in which one can smoothly embed $\Sigma$ into $M$. Naively  is, we are interested in the space
            \begin{equation}
                E \equiv \{ \text{smooth embeddings } e: \Sigma \ to M \}
            \end{equation}

            But note that this really isn't quite what we're looking for. This is because one can pre-compose an embedding $e$ with an automorphism $f: \Sigma \to \Sigma$. The composed map $e \circ f \in E$ is another smooth embedding of $\Sigma$ in $M$, but their images are identical

            \begin{equation}
                e(\Sigma) = e(f(\Sigma)) \subset M
            \end{equation}

            So they really describe the same embedding of $\Sigma$ in $M$. So in order to define a less redundant notion of ``the space of all embeddings'', we need to quotient our space by these precompositions

            \begin{equation}
                \tilde E \equiv E/\sim ~,~ e_1 \sim e_2 \iff e_2 = e_1 \circ f \text{ for some } f \in Aut(\Sigma)
            \end{equation}

            This is the notion of the moduli space in the classical theory of a bosonic string.

        \paragraph{The space of induced metrics} 

            It is also useful to consider the space of induced metrics.

            \begin{equation}
                \tilde F \equiv \{e^* g_M | e \in \tilde E \} \subset T\Sigma^{\otimes 2}
            \end{equation}

            Recall we obtained $\tilde E$ as a quotient. It is useful to know how this quotient structure manifests in the space $\tilde F$. Recall that the de Rham functor is contravariant

            \begin{equation}
                (e \circ f)^* = f^* \circ e^*
            \end{equation}

            Hence we can write

            \begin{align}
                F &\equiv \{ e^*g_M | e \in E \}\\
                \tilde F &\equiv F / \sim ~,~ h_1 \sim h_2 \iff h_1 = f^* h_2 \text{ for some } f \in Aut(\Sigma)
            \end{align}

        \subsection{Gauge Fixing}
            \paragraph{The general story}

                Recall again that our moduli space has is a quotient

                \begin{equation}
                    \tilde E = E / \sim
                \end{equation}

                Gauge fixing means smoothly choosing a representative for each equivalence class in this quotient In other words, we want a smooth section $s$ with respect to the quotient map $\pi: E \to \tilde E$.

                The reason we want to do this is to be able to explicitly perform the variational problem for the functional $S[e]$. $S[e]$ is straightforward to write down as a problem with the domain being $E$. In order to reduce the variational problem to the quotiented domain of maps $\tilde E$, one must choose representatives from each equivalence class $[e]$. We would also like this choice of representatives to be smooth, so that one can actually consider the variations of these representatives.

                Given a ``gauge choice'' $s: \tilde E \to E$ as described above, one can simply considers the variational problem with the domain $s(\tilde E)$.


        \section{Actions}

            \subsection{Nambu-Goto Action}
                Define the Nambu-Goto action as

                \begin{equation}
                    S_{NG} \in C^\infty(\tilde E), ~~ [e] \mapsto \int_\Sigma \omega_{e^*g_M}
                \end{equation}

                where $\omega_g$ denote the canonical Riemannian volume form associated with (pseudo) Riemannian metric $g$. Note that this is well-defined. Suppose $e' \sim e$, then $e' = e \circ f$ where $f$ is some automorphism, then

                \begin{equation}
                    S_{NG}[e'] = \int_\Sigma \omega_{e'^*g_M} = \int_\Sigma \omega_{f^*e^*g_M} = \int_\Sigma f^* \omega_{e^* g_M} = \int_{f^{-1}(\Sigma) = \Sigma} \omega_{e^* g_M} = S_{NG}[e]
                \end{equation}

                where we used the fact that under a diffeomorphism, the volume form obeys $f^*\omega_{g} = \omega_{f^*g}$

            \subsection{Polyakov Action}
                We now consider the following functional.

                \begin{align}
                    S_P: \tilde E \times Met(\Sigma) \to \mathbb R\\
                    ([e], h) \mapsto \int_\Sigma \langle h^{-1}, e^*g_M \rangle\omega_h  
                \end{align}

                where $Met(\Sigma)$ denotes the space of all signature $(1, 1)$ metrics on $\Sigma$, and $\omega_h$ is again the associated Riemannian volume form. i.e. all smooth symmetric rank-(2, 0) tensor fields satisfying the pseudo-riemannian condition. $h^{-1} \in T\Sigma^{\otimes 2}$ is the inverse metric, and $\langle \cdot, \cdot \rangle$ is the usual contraction using the metric inner product, in this case a map $T\Sigma^{\otimes 2} \otimes T\Sigma^{*\otimes 2} \to \mathbb R$.

                The Polyakov action is related to the Nambu action given above very simply. Observe

                \begin{equation}
                    S_P[e, e^*g_M] = \int_\Sigma \langle (e^*g_M)^{-1}, e^*g_M \rangle\omega_{e^*g_M}   =  S_{NG}[e]
                \end{equation}

                One can verify the following lemma through a brute force computation

                \begin{equation}
                    \frac{\partial S_P[e, h]}{\partial h} = 0 \implies h = e^*g_M
                \end{equation}

                These two statements together says that one can perform the variation problem for $S_P$ in the argument $h$, substitude the result back in $S_P$, and retrieve $S_{NG}$.


\end{document}
